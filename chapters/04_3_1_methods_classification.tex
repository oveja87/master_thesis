\subsection{Classification} 
\label{section:classificationMethod}

%The classification system is build on basis of the analysis of the different strokes frequency spectra and the existing easydrum e-drum configurator. Thereby it is attached importance to the usability for the final system. Consequently, the training may not take too much time for the user. The developed method needs a maximum of four strokes of each drum for training.
%
%For feature extraction and classification, there are developed different methods. The first method extracts single feature values as a feature vector and uses them for the classification with a classification algorithm. There are tested different classification algorithms, whereas the best results are gained by decision trees and Naive Bayes. The classification algorithms are applied with the help of Weka. The second method creates templates out of the frequency spectrum of a drum stroke, which are used to compare with in the classification part. Thereby each bin of a isolated frequency bin is considered. It is also tried to use only the frequency bins at frequency peak positions.